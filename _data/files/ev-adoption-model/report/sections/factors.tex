\section{Adoption Factors} \label{section:factors}
As mentioned in section \ref{section:abstract}, this study is consumer-centred, which means that the study of adoption of electric vehicles is performed taking into consideration the factors that the consumer is more "passionate" about when choosing an electric vehicle. Therefore, the logic way to find these adoption factors is to base our selection on consumer-focused surveys. Particularly, the selection of factors that were chosen to this study were based on the information gathered from the surveys referenced in \cite{thesis-base, yue-xiang-paper}.

Having said that, since the model was mainly based on \cite{thesis-base}, the adoption factors included are very similar with the base model, with a few additional ones that I found to be as significant as the ones already taken into consideration. Thus, as referenced in \cite{thesis-base}, there are four technical factors that have the greatest impact on the consumer decision: \textbf{driving range}, \textbf{price}, \textbf{recharge infrastructure} and \textbf{recharging time}. In addition to these factors, we must also include the \textbf{effectiveness of advertising} such vehicles, the \textbf{word of mouth effect} and last, but not least, the \textbf{environmental concern effect}.

\subsection{Driving Range}
The maximum driving range of an EV is mostly related to the performance of the battery. Nowadays, the improvements in lithium-ion battery technology allows driving ranges on average around 200km - 300km. However, following current events, there have been some interesting developments in the field, promoted by Tesla and what currently is one of the downsides of EVs (when compared to the autonomy values of fossil fuel cars) could, in the near future, be a major advantage.

\subsection{Price}
The current purchase price of an electric vehicle is, in general, twice that of a comparable fossil fuel car (e.g. gasoline or diesel). With the mass production of more and more models, this price will eventually go down, but how fast this will happen is the difference between a more quick or slow mass adoption of EVs, which actually is kind of a paradox.

The problem with this price point is that the general consumer is not willing to pay this amount of money for an EV in contrast with governments and fleet operators that are more open to these types of purchases. However, one key factor that has been playing a big role as an incentive to the general consumer is the benefit most governments have been allowing when buying an EV, such as subsidies and/or tax reduction.

\subsubsection{Fixed Price}
The high purchase price of an EV is mostly due to the high battery prices and low production volumes. However, this premium cost for EVs is still compensated by the low cost of electricity compared to gasoline. The predictions state that, over time, the electric vehicle costs will decrease as the result of ongoing R\&D, learning effect and mass production.

\subsubsection{Variable Price}
EVs, in addition to benefiting from the lower cost of electricity when compared to fossil fuels, are also influenced by other "external" factors that do have an impact on the variable costs difference between an EV and fossil fuel vehicles, such as the rising oil prices which could make ICVs less attractive and lower maintenance costs. However, battery devaluation and the effect of environmental temperature have their own impact too, being some of the greatest disadvantages of EVs. Furthermore, there are subsidies that should stimulate the early adoption of electric vehicles.

Simplifying, the variable costs depend on a set of different values: the \textbf{influence of oil} and \textbf{electricity price}, \textbf{subsidies}, \textbf{maintenance costs}, \textbf{R\&D influences} and \textbf{devaluation of the battery}. Finally, among all these variable costs, the subsidies are the best values to influence and manipulate in simulation scenarios.

\subsection{Recharge Infrastructure}
Availability of charging infrastructure is essential to the market adoption of electric vehicles. Furthermore, the different charging alternatives, for instance the fast charging stations and the infrastructure density have a big influence on the electric vehicle's range and charging times, which additionally affects the possible market share. Which this means is that a high density of infrastructure charging locations (and fast recharge times) makes more acceptable a lower maximum driving range in comparison with a low density and slow recharge times.

\subsection{Recharging Time}
Overall, an electric vehicle requires a battery capacity of 65kWh or more to drive a 300km range. This means that it will take 5 - 6,5 hours to recharge the EV in one's garage. In the case of three phase charge station, that are more common in large homes, offices and firms, it will take approximately 1 - 1,5 hours to fully recharge the vehicle.

Besides the two types of recharging possibilities, there is also another which is mentioned as a fast charging station, despite being a lot faster, has some disadvantages because they are too expensive, making it much more difficult (almost impossible, at least right now) to include this type of research into the electric vehicle.

\subsection{Effectiveness of Advertising}
The adoption from advertisement depends on the advertising effect. This means that, depending on how effective the advertisements are, the rate of adoption will be faster or slower.

The advertising effect works in a way that the adopters made the decision on buying an EV solely on the advertising strategy (websites, pilot projects, commercials, etc.) and the major obstacle to the success of this effect is the newness of EVs that keeps potential adopters from buying one.

\subsection{Word of Mouth Effect}
The adoption by word of mouth is estimated taking into consideration Portugal's population, a contact rate and the adoption fraction that will be further explained in the next section.

This factor creates an upper limit for the adoption model and has a dissemination behaviour similar to a contagious disease, that is, the greater the number of people adopting EVs and talking about their potentially positive experience, the greater the effectiveness of the word of mouth effect. This means that in early stages of EVs market introduction this is a minor factor, but over time, becomes a much bigger influence on the market growth.

\subsection{Environmental Concern Effect}
The last factor is selected over the observation that more and more people have a very real concern about our planet and environment and being the fossil fuels industry one of biggest responsible for worldwide pollution, this factor is more and more real and is supposed to have a big impact on the EV market in the near future.

\clearpage