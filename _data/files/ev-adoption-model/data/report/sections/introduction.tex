\section{Introductory Note} \label{section:intro}
Recently, with hard evidences of the Global Warming problem, every industry has become more concerned with its environmental impact on the world. This concern is a big motivator to innovate current technologies to be more clean and use renewable energy sources, to promote a sustainable growth with a less polluting behaviour.

One of these industries is the car industry. According to \cite{Weiss00onthe} and \cite{eu-parliament}, in 2019, the transportation sector was responsible for almost 30\% of the total CO2 emissions in European Union, and nearly 20\% worldwide, which means that the use of electric vehicles is a necessity to our planet's health and also means that it is an in-development market with a remarkable on-going year-over-year (YoY) growth.

Hence, so I could have a more clear picture of how the EV market would evolve in Portugal, I decided to study the adoption process of electric vehicles based on Portuguese market indicators. This document is the result of that study and provides an explanation on the process of adoption of electric vehicles as a personal one, taking into account current and future: population, adoption ratio, price difference, battery range and other factors that are essential to the decision making process of choosing an EV as a personal vehicle.

The developed dynamic model was implemented using Vensim \cite{vensim, mod-helper, vensim-youtube}, based on the different works of Pedro Ferreira, Yue Xiang \textit{et. al} and Tim Bongard \cite{thesis-base, pedro-report, yue-xiang-paper}, respectively.

\clearpage
