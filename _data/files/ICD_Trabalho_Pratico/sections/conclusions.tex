\section{Conclusões}

No paradigma da evolução das nossas capacidades como estudantes de engenharia informática, a realização deste projeto permitiu-nos consolidar a aprendizagem da UC de Infraestutura de Centro de Dados, mais concretamente, os métodos e ferramentas que devem ser aplicados no que toca à resolução do problema de planear, analisar e operar uma infraestrutura de elevada disponibilidade e desempenho.

O planeamento deste tipo de infraestruturas é uma das fases essenciais para que a aplicação seja capaz de corresponder a todas as exigências capazes de surgirem. Sendo ainda mais importante, quando estamos a planear uma tipologia que servirá como base de um serviço de grande escala, devido ao requisito de escalabilidade.

Apesar de considerarmos que tivemos sucesso na implementação deste tipo de arquitetura, consideramos que existem aspetos nos quais poderíamos melhorar em iterações futuras, nomeadamente a diversificação de testes realizados, de modo a obter mais cenários de uso, e, a execução de uma maior quantidade de testes e modo a obter valores mais precisos.

\iffalse
Começámos o projeto com a criação de uma arquitetura básica, constituída por apenas duas máquinas conectadas entre si. Numa das máquinas instalamos a aplicação \textbf{wiki.js} e na outra uma base de dados \textbf{postgreSQL}. Esta implementação, evidentemente, não foi capaz de corresponder aos requisitos exigidos. A posterior realização e análise de testes de carga, a esta tipologia, veio provar isso mesmo.

Depois, utilizando as tecnologias lecionadas, idealizámos e instalámos uma tipologia que teoricamente seria capaz de corresponder aos nossos objetivos. De todos os métodos que acabámos por utilizar, achamos importante salientar os \textbf{load balancers} e os \textbf{clusters} que nos permitiram eliminar por completo os \textit{bottlenecks} existentes na versão anterior.

Os testes de carga que realizámos sobre a segunda versão, vieram demonstrar que houve uma tremenda evolução na capacidade de resposta e permitiram-nos concluir que esta implementação garante elevada disponibilidade. Porém, achamos que a nossa base de dados, apesar de todas as melhorias de performance, continua a ser um \textit{bottleneck} e que a sua implementação podia ser melhorada. 

Por fim, a nossa arquitetura satisfaz os requisitos no contexto do trabalho, pelo que, não consideramos necessário alterar a implementação da base de dados.
\fi 


\pagebreak